\documentclass[12pt]{article}
\usepackage{graphicx} % Required for inserting images
\usepackage[letter, margin=0.5in]{geometry}
\usepackage{amsmath, mathrsfs, amssymb, cancel} % for various mathematical symbols

\graphicspath{{./images/}}

\setcounter{secnumdepth}{0} % Removes section numbering

\usepackage{titling}
\renewcommand\maketitlehooka{\null\mbox{}\vfill}
\renewcommand\maketitlehookd{\vfill\null}

\title{Physical Mechanics Homework 6}
\date{10/17/2024}
\author{Damien Koon}



\begin{document}

\maketitle

\newpage
\textbf{
(1). \newline
(a). Derive the expression for the energy of a damped oscillator as a function of time. \newline
(b). Derive the expression for the energy loss rate of a damped oscillator as a function
of time. \newline
(c). For a lightly damped oscillator, calculate the average rate at which the oscillator
loses energy (i.e. compute a time average over one cycle). \newline
}

From the differential equation:
$$
m \ddot{x} + c \dot{x} + k x = 0
$$

With $\beta = \frac{c}{2m}$ and $\omega_0^2 = \frac{k}{m}$

$$
\ddot{x} + 2 \beta \dot{x} + \omega_0^2 x = 0
$$

Which turns into the auxiliary equation:
$$
\alpha^2 + 2 \beta \alpha + \omega_0^2 = 0
$$

Which gives the solution to the auxiliary equation as:
$$
\alpha = - \beta \pm \sqrt{\beta^2 - \omega_0^2}
$$

From (c), it's clear that the oscillation is underdamped, indicating that $\beta  < \omega_0$
$$
\alpha = - \beta \pm i \sqrt{\omega_0^2 - \beta^2} = - \beta \pm \omega_1
$$

With $\omega_1 = \sqrt{\omega_0^2 - \beta^2}$ the solution is 

$$
x(t) = A e^{- \beta t} cos(\omega_1 t + \phi)
$$

$$
\dot{x}(t) = - A \beta e^{- \beta t} cos(\omega_1 t + \phi) - \omega_1 A e^{- \beta t} sin(\omega_1 t + \phi)
$$

With $p = \omega_1 t + \phi$
$$
x^2(t) = A^2 e^{-2 \beta t} cos^2(p)
$$

$$
\dot{x}^2(t) = A^2 \beta^2 e^{-2 \beta t} cos^2(p) + A^2 \omega_1^2 e^{-2 \beta t} sin^2(p) + 2 A^2 \beta \omega_1 cos(p) sin(p)
$$

$$
E = T + U = \frac{1}{2} m \dot{x}^2 + \frac{1}{2}k x^2 = \frac{1}{2}m(\dot{x}^2 + \omega_0^2 x)
$$

Inputting $x^2(t)$ and $\dot{x}^2(t)$ into E,
$$
E = \frac{1}{2} m A^2 e^{-2 \beta t}( \beta^2 cos^2(p) + \omega_1^2 sin^2(p) + 2 \beta \omega_1 cos(p) sin(p) + \omega_0^2 cos^2(p))
$$
\newpage
$$
E = \frac{1}{2} m A^2 e^{-2 \beta t} ( (\beta^2 + \omega_0^2)cos^2(p) + (\omega_0^2 - \beta^2)sin^2(p) + \beta \omega_1 sin(2p))
$$

$$
E = \frac{1}{2} m A^2 e^{-2 \beta t} ( \beta^2 (cos^2(p) - sin^2(p)) + \omega_0^2 ( sin^2(p) + cos^2(p) ) + \beta \omega_1 sin(2p) )
$$

$$
E(t) = \frac{1}{2} m A^2 e^{-2 \beta t} ( \beta^2 cos(2p) + \beta \omega_1 sin(2p) + \omega_0^2 )
$$

$$
\dot{E} = - m A^2 \beta e^{-2 \beta t} ( \beta^2 cos(2p) + \beta \omega_1 sin(2p) + \omega_0^2 ) + \frac{1}{2} m A^2 e^{-2 \beta t} (-2 \beta^2 \omega_1 sin(2p) + 2 \beta \omega_1^2 cos(2p) )
$$

$$
\dot{E} = - m A^2 \beta e^{-2 \beta t} ( \beta^2 cos(2p) + \beta \omega_1 sin(2p) + \omega_0^2 + \beta \omega_1 sin(2p) - \omega_1^2 cos(2p) )
$$

$$
\dot{E} = - m A^2 \beta e^{-2 \beta t} (( \beta^2 - \omega_1^2 )cos(2p) + 2 \beta \omega_1 sin(2p) + \omega_0^2 ) 
$$

Let $\beta << \omega_0$

$$
\omega_1 = \sqrt{\omega_0^2 - \beta^2} = \omega_0 \sqrt{1 - \frac{\beta^2}{\omega_0^2}}
$$

For small $\beta$, 
$$
\beta^2 \approx 0 \therefore \omega_1 = \omega_0
$$

From this, the sin(2p) and cos(2p) terms go to 0, and the only term remaining is the $\omega_0^2$ term
$$
\dot{E} \approx - m A^2 \beta \omega_0^2 e^{-2 \beta t}
$$



\newpage
\textbf{
(2). The amplitude of a weakly-damped oscillator (i.e. $\beta << 1$)
decreases to $\frac{1}{e}$ of its initial value after n complete cycles. Show that the ratio of the period of
the oscillator to the period of the same oscillator with no damping is given by
}

$$
\frac{\tau_1}{\tau_0} \approx \sqrt{1+\frac{1}{4 \pi^2 n^2}}
$$

From Newton's 2nd law,
$$
m\ddot{x} + c \dot{x} + k x = 0
$$
$$
\beta = \frac{c}{2m} \hspace{1cm} \omega_0^2 = \frac{k}{m}
$$

$$
\ddot{x} + 2 \beta \dot{x} + \omega_0^2 x = 0
$$

With the auxiliary equation and solutions, 
$$
\alpha^2 + 2 \beta \alpha + \omega_0^2 = 0
$$

$$
\alpha = -\beta \pm \sqrt{\beta^2 - \omega^2_0}
$$
$$
\alpha = -\beta \pm i\sqrt{\omega_0^2 - \beta^2_0}
$$

With $\omega_1 = \sqrt{\omega_0^2 - \beta^2}$
$$
\alpha = - \beta \pm i \omega_1
$$


Therefore the solution to the differential equation is,
$$
x(t) = Ae^{-\beta t} cos(\omega_1 t + \phi)
$$

Where the amplitude,
$$
A(t) = A_0 e^{- \beta t}
$$

$\matchscr{L}$et $t = n\tau_0$

$$
A(t_n) = \frac{A_0}{e} = A_0 e^{- \beta \taU_0} \rightarrow \frac{1}{e} = e^{- \beta \tau_0}
$$

$$
n \beta \tau_0 = 1 \implies \beta = \frac{1}{n \tau_0} = \frac{\omega_0}{2 \pi n}
$$

$$
\omega^2_1 = \omega^2_0 - \beta^2 \implies \omega^2_0 = \omega^2_1 + \beta^2
$$

Using the previous definition of $\beta$
$$
\omega_0 = \sqrt{\omega_1^2 + \frac{\omega_1^2}{4 \pi^2 n^2}} = \omega_1 \sqrt{1 + \frac{1}{4 \pi^2 n^2}}
$$

$$
\frac{\tau_1}{\tau_0} = \frac{\omega_0}{\omega_1} = \cancel{\frac{\omega_1}{\omega_1}} \sqrt{1 + \frac{1}{4 \pi^2 n^2}}
$$
Using the Taylor series approximation of $\sqrt{1+\frac{1}{x}} = 1 + \frac{1}{2x}$ and for $n >> 1$

$$
\boldsymbol{
\frac{\tau_1}{\tau_0} = 1 + \frac{1}{8 \pi^2 n^2}
}
$$

\newpage
\textbf{
(3). Show that} $x(t) = (A + Bt)e^{-\beta t}$ \textbf{ is indeed the solution for critical damping by assuming a solution of the form} $x(t) = y(t) e^{-\beta t}$ \textbf{ and determining the function} $y(t)$

\hspace{0.5cm}

From $x(t) = y(t) e^{- \beta t}$

$$
x = y e^{-\beta t}
$$

$$
\dot{x} = \dot{y} e^{- \beta t} - y \beta e^{- \beta t}
$$

$$
\Ddot{x} = \ddot{y} e^{- \beta t} - \beta \dot{y} e^{- \beta t} - \beta \dot{y} e^{- \beta t} + y \beta^2 e^{- \beta t}
$$

$$
\ddot{x} = \ddot{y} e^{- \beta t} - 2 \beta \dot{y} e^{- \beta t} + y \beta^2 e^{- \beta t}
$$

For a critically damped oscillator,
$$
\beta = \frac{c}{2m} \hspace{2cm} \omega_0^2 = \frac{k}{m} = \beta^2
$$

From Newton's 2nd law in differential form,
$$
m \Ddot{x} + c \dot{x} + kx = 0
$$

$$
\Ddot{x} + 2 \beta \dot{x} + \omega_0^2 x = 0
$$

$$
\Ddot{x} + 2 \beta \dot{x} + \beta^2 x = 0
$$

Inputting $x$, $\dot{x}$, and $\ddot{x}$ into the differential equation, 
$$
e^{-\beta t} ( \ddot{y} - 2 \dot{y} \beta + y \beta^2 + 2 \dot{y} \beta - 2 \beta^2 y + \beta^2 y ) = 0
$$

After reducing, the expression becomes 

$$
e^{- \beta t} \ddot{y} = 0 \implies \ddot{y} = 0
$$

After integrating,
$$
y(t) = A + Bt
$$

Therefore, 
$$
\boldsymbol{x(t) = (A + Bt) e^{-\beta t}}
$$



\newpage
\textbf{
(4). A grandfather clock has a pendulum length of 0.7 m (with ignorable mass)
and a bob at the end of mass 0.4 kg. A second mass of 2 kg falls 0.8 m in seven days to keep
the amplitude of the pendulum’s oscillation steady at 0.03 rad. What is the quality factor Q of the system?
}

$\mathscr{L}$et A = 0.03 rad, $m_1 = 0.4 kg$, $\ell_1 = 0.7m$, $m_2 = 2 kg$, $\ell_2 = 0.8m$

\vspace{0.5cm}

Consider the pendulum of the grandfather clock to be undamped,

$$
m \ddot{x} = m \ell \ddot{\theta}
$$

$$
m \ell \ddot{\theta} + mg sin(\theta) = 0
$$

For very small $\theta$, assume $sin(\theta) \approx \theta$,

$$
\ell \ddot{\theta} + g \theta = 0
$$

With $\frac{g}{\ell} = \omega_0^2$
$$
\ddot{\theta} + \omega_0^2 \theta = 0
$$

Which has the solution,
$$
\theta(t) = A cos(\omega_0 t + \phi)
$$

Which follows that,
$$
\dot{\theta}(t) = -A \omega_0 sin(\omega_0 t + \phi)
$$
$$
\theta^2(t) = A^2 cos^2(\omega_0 t + \phi)
$$
$$
\dot{\theta}^2(t) = A^2 \omega_0^2 sin^2(\omega_0 t + \phi)
$$

From this,
$$
E = \frac{1}{2} m \dot{x}^2 + \frac{1}{2}kx^2 = \frac{1}{2}m(\dot{x}^2 + \omega_0^2 x^2) = \frac{1}{2} m \ell^2 (\dot{\theta} + \omega_0^2 \theta)
$$
$$
E = \frac{1}{2} m \ell^2 \omega_0^2 A^2 ( sin^2(\omega_0 t + \phi) + cos^2(\omega_0 t + \phi) ) = \frac{1}{2} m \ell^2 \omega_0^2 A^2 = \frac{1}{2} m A^2 \ell^2 \frac{g}{\ell} = \frac{1}{2} m A^2 \ell g
$$

An alternate solution to E follows as such,

Assume that the pendulum is at the end of its period. 
Thus, the pendulum has max potential energy and no kinetic energy

The height of the mass at the end of the pendulum is
$$
h = (\ell_1 - \ell_1 cos(A)
$$

$$
U = m_1 g h = m_1 g (\ell_1 - \ell_1 cos(0.03 rad))
$$

To find the max value of kinetic energy, use the velocity, $v = \frac{distance}{time} = \frac{\ell_1}{7 days}$

$$
K = \frac{1}{2} m_1 v^2 = \frac{1}{2} m_1 (\frac{\ell_1}{7 days})^2
$$

$$
E = K + U = \frac{1}{2} m_1 (\frac{\ell_1}{7 days})^2 + m_1 g (\ell_1 - \ell_1 cos(0.03 rad))
$$

Both of these values of E are equivalent. The initial value will be used.

To find the number of cycles of the pendulum over 7 days,
$$
\tau = 2 \pi \sqrt{\frac{\ell_1}{g}} = 2 \pi \sqrt{\frac{0.7 m}{ 9.81 \frac{m}{s^2}}} = \frac{1.68 s}{cycle}
$$

$$
\frac{7 days}{1.68s} = \frac{7\hspace{0.1cm} days * 24\hspace{0.1cm} hours * 3600 s}{1.68 s} = 360000\hspace{0.1cm} cycles
$$

Considering the total energy of the system of the 2nd mass and length,
$$
\Delta E = \frac{energy}{cycles} = \frac{m_2 g \ell_2}{360000} = 4.36 \times 10^{-5} J
$$

$$
Q = 2 \pi \frac{E}{\Delta E} = 2 \pi \frac{\frac{1}{2} m_1 A^2 \ell g}{4.36 \times 10^{-5}} = 178
$$

\newpage
\textbf{
(5). An overdamped oscillator can overshoot x = 0 before eventually returning to
that position. \newline
(a). Show that in order for this to occur, the initial displacement $x_0$ and initial velocity $v_0$ must satisfy the condition \newline
$$
v_0 = x_0 [-\beta - \frac{\omega_2}{tanh(\omega_2 t)}]
$$
(b). Assuming that both $\beta$ and $\omega_2$ are positive real numbers, what conclusion can you
draw about the vector quantities (i.e. magnitude and direction) $x_0$ and $v_0$ \newline
}



From the differential equation:
$$
m \ddot{x} + c \dot{x} + k x = 0
$$

With $\beta = \frac{c}{2m}$ and $\omega_0^2 = \frac{k}{m}$

$$
\ddot{x} + 2 \beta \dot{x} + \omega_0^2 x = 0
$$

which turns into the auxiliary equation:

$$
\alpha^2 + 2 \beta \alpha + \omega_0^2 = 0
$$

With $\omega_2 = \sqrt{\beta^2 - \omega_0^2}$
$$
\alpha = - \beta \pm \omega_2
$$

Which produces the solution
$$
x(t) = A_1 e^{-(\beta - \omega_2) t} + A_2 e^{-(\beta + \omega_2) t}
$$

Which follows that,
$$
\dot{x}(t) = -A_1 (\beta - \omega_2) e^{-(\beta - \omega_2) t} - A_2 (\beta + \omega_2) e^{-(\beta + \omega_2) t}
$$

$$
x(0) = A_1 + A_2
$$

$$
\dot{x}(0) = - A_1 ( \beta - \omega_2 ) - A_2 ( \beta + \omega_2 )
$$

From the definitions of $x(0) = x_0$ and $\dot{x}(0) = v_0$, the definitions of $A_1$ and $A_2$ follow as such,

$$
A_1 = \frac{x_0 ( \beta + \omega_2 ) + v_0}{2 \omega_2}
$$
$$
A_2 = -\frac{x_0 ( \beta - \omega_2 ) + v_0}{2 \omega_2}
$$


$\mathscr{L}$et x(t) = 0,
$$
A_1 e^{-(\beta - \omega_2) t} + A_2 e^{-(\beta + \omega_2) t} = 0 
$$

$$
A_1 e^{-(\beta - \omega_2) t} = - A_2 e^{-(\beta + \omega_2) t}
$$

$$
- \frac{A_2}{A_1} = \frac{e^{-(\beta - \omega_2) t}}{e^{-(\beta + \omega_2) t}} = e^{2 \omega_2 t}
$$

$$
e^{2 \omega_2 t} = - \frac{A_2}{A_1} = \frac{x_0 ( \beta - \omega_2 ) + v_0}{x_0 ( \beta + \omega_2 ) + v_0}
$$

From this definition and some algebra,

$$
x_0 ( \beta - \omega_2 ) + v_0 = e^{2 \omega_2 t} ( x_0 ( \beta + \omega_2 ) + v_0 )
$$

$$
x_0 ( \beta - \omega_2 - e^{2 \omega_2 t} (\beta + \omega_2) ) = v_0 ( e^{2 \omega_2 t} - 1 )
$$

$$
x_0 ( \beta ( 1 - e^{2 \omega_2 t} ) - \omega_2 (1 + e^{2 \omega_2 t} ) ) = v_0 ( e^{2 \omega_2 t} - 1 )
$$

$$
x_0 ( - \beta - \omega_2 (\frac{1 + e^{2 \omega_2 t}}{e^{2 \omega_2 t} -1})) = v_0
$$

Thus,
$$
v_0 = x_0 [-\beta - \frac{\omega_2}{tanh(\omega_2 t)}]
$$

$\mathscr{L}et$ $x_0$, $v_0$ \in \mathbb{R} > 0

From the definition of $v_0$,

$$
- v_0 = x_0 [\beta \frac{\omega_2}{tanh(\omega_2 t)}]
$$

Thus, the magnitude of $x_0$ and $v_0$ is
$$
x_0 > 0 \hspace{0.5cm} v_0 < 0
$$

The direction of $v_0$ is always toward the origin, and the direction of $x_0$ is oppisite to $v_0$


\end{document}