\documentclass[12pt]{article}
\usepackage{graphicx} % Required for inserting images
\usepackage[letter, margin=0.5in]{geometry}
\usepackage{amsmath, mathrsfs, amssymb} % for various mathematical symbols

\graphicspath{{./images/}}

\setcounter{secnumdepth}{0} % Removes section numbering

\title{Physical Mechanics Homework 5}
\date{10/01/2024}
\author{Damien Koon}

\begin{document}

\maketitle

\newpage
\textbf{
(1). Two masses $m_1 = 100 g$ and $m_2 = 200 g$ slide freely on a horizontal frictionless track and are connected by a spring whose force constant is $k = 0.5 N/m.$ Find the frequency of oscillatory motion for this system. (Hint: assume that the natural length of the spring is $\ell$. How does that relate to the positions of the masses at any given time?)
}

$$\Vec{F} = m \ddot{x} = -kx$$

$$
m_1 \ddot{x_1} = -k ( x_1 - x_2 + \ell)
$$

$$
m_2 \ddot{x_2} = -k ( x_2 - x_1 - \ell) \rightarrow x_1 = \frac{m_2 \ddot{x_2} + k x_2 - k \ell}{k}
$$

Substituting $x_1$ into the first equation for $x_1$ gives
$$
m_1 \ddot{x_1} = -k(x_2 -x_2 + \ell - \ell) - m_2 \ddot{x_2} \rightarrow m_1 \ddot{x_1} = - m_2 \ddot{x_2}
$$

$$
x_1 = \frac{m_2 \ddot{x_2} + k x_2 - k \ell}{k} \implies \ddot{x_1} = \frac{d^2}{dt^2}(\frac{m_2 \ddot{x_2} + k x_2 - k \ell}{k})
$$

Substituting $\ddot{x_1}$ into $m_1 \ddot{x_1} = - m_2 \ddot{x_2}$ gives

$$
m_1 \frac{d^2}{dt^2}(\frac{m_2 \ddot{x_2} + k x_2 - k \ell}{k}) = - m_2 \ddot{x_2}
$$

$m_1,m_2, k$, and $\ell$ are constants, thus the equation can be rewritten as

$$
\frac{m_1 m_2}{k} \frac{d^2}{dt^2}(\ddot{x_2}) + m_1 \ddot{x_2} + m_2 \ddot{x_2} = 0
$$

$$
\frac{d^2}{dt^2}(\ddot{x_2}(\frac{m_1 m_2}{k}) + {x_2}(m_1 + m_2)) = 0 \implies \ddot{x_2}(\frac{m_1 m_2}{k}) + {x_2}(m_1 + m_2) = 0
$$

From this, the equation can be written into the form of the differential equation 

$$\ddot{x} + \omega_0^2 x = 0 \rightarrow \ddot{x} + \frac{k}{m} x = 0$$

with m being the sum of the inverse masses 
$$
\frac{1}{m} = \frac{1}{m_1} + \frac{1}{m_2} = \frac{m_1 + m_2}{m_1 m_2}
$$

$$
\therefore \omega^2 = \frac{k(m_1 + m_2)}{m_1 m_2} \implies \omega = \sqrt{\frac{k(m_1 + m_2)}{m_1 m_2}} 
$$

$$
\boldsymbol{\omega} = \sqrt{\frac{0.5 \frac{N}{m}(0.1 kg + 0.2 kg)}{0.1 kg * 0.2 kg}} = \boldsymbol{2.74 Hz}
$$

\newpage
\textbf{
(2). On the surface of the Moon, the acceleration of gravity is about one sixth that on the surface of Earth. What is the period of a simple pendulum of length $1 m$ on the Moon?
}

$$
F = m \ddot{x} = - mg sin(\theta)
$$

$$
\ddot{x} = -g sin(\theta)
$$

Using $\ell \theta = s \implies \ell \ddot{\theta} = \ddot{s}$

$$
\ddot{s} + g sin(\theta) = 0
$$

$$
\ell \ddot{\theta} + gsin(\theta) = 0
$$

For $\theta < 20^{\circ}, sin(\theta) \approx \theta$ thus,

$$
\ddot{\theta} + \frac{g}{\ell} \theta = 0
$$

This differential equation can be mapped to $\ddot{x} + \omega^2 x = 0$, thus

$$
\ddot{\theta} + \frac{g}{\ell} \theta = 0 \Longleftrightarrow \ddot{x} + \omega^2 x = 0
$$

$$\omega^2 = \frac{g}{\ell}$$



$$
\omega_0 = \sqrt{\frac{g}{\ell}}
$$

$$
T = 2 \pi \sqrt{\frac{\ell}{g}}
$$

$$
\boldsymbol{T} = 2 \pi \sqrt{\frac{1m}{9.81 \frac{m}{s^2} / 6}} = \boldsymbol{4.91 s}
$$
\newpage

\textbf{
(3). Two springs with stiffnesses $k_1$ and $k_2$ are used in a vertical position to support a single object of mass $m$.}
\newline

\textbf{
(a). Show that the angular speed of oscillation is $\omega_0 = \sqrt{\frac{k_1 + k_2}{m}}$ if the springs are connected in parallel.
}
\newline

\textbf{
(b). Show that the angular speed of oscillation is $\omega_0 = \sqrt{\frac{k_1 k_2}{(k_1 + k_2)m}}$ if the springs are connected in series.
} \newline 

For a system of springs connected in parallel, $k_{eq} = k_1 + k_2$, thus

$$
F = m \ddot{x} = -k x = -x (k_1 + k_2)
$$

$$
\ddot{x} + x (\frac{k_1 + k_2}{m}) = 0
$$

Which is equivalent to the differential equation

$$
\ddot{x} + x (\frac{k_1 + k_2}{m}) = 0 \Longleftrightarrow \ddot{x} + \omega_0^2 x = 0
$$

$$
\therefore \omega_0^2 = \frac{k_1 + k_2}{m} \implies \boldsymbol{\omega_0 = \sqrt{\frac{k_1 + k_2}{m}}}
$$

For a system of springs connected in series, $\frac{1}{k_{eq}} = \frac{1}{k_1} + \frac{1}{k_2}$

$$
\frac{1}{k_{eq}} = \frac{1}{k_1} + \frac{1}{k_2} \implies k_{eq} = \frac{k_1 k_2}{k_1 + k_2}
$$

$$
F = m \ddot{x} = -k x = - x (\frac{k_1 k_2}{k_1 + k_2})
$$

$$
m \ddot{x} + x (\frac{k_1 k_2}{k_1 + k_2}) = 0
$$

$$
\ddot{x} + x (\frac{k_1 k_2}{m(k_1 + k_2)}) = 0
$$

This equation is equivalent to the differential equation

$$
\ddot{x} + x (\frac{k_1 k_2}{m(k_1 + k_2)}) = 0 \Longleftrightarrow \ddot{x} + \omega_0^2 x = 0
$$

This implies that
$$
\omega_0^2 = \frac{k_1 k_2}{m(k_1 + k_2)} \implies \boldsymbol{\omega_0 = \sqrt{\frac{k_1 k_2}{m(k_1 + k_2)}}}
$$
\end{document}